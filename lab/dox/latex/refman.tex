\documentclass{book}
\usepackage[a4paper,top=2.5cm,bottom=2.5cm,left=2.5cm,right=2.5cm]{geometry}
\usepackage{makeidx}
\usepackage{natbib}
\usepackage{graphicx}
\usepackage{multicol}
\usepackage{float}
\usepackage{listings}
\usepackage{color}
\usepackage{ifthen}
\usepackage[table]{xcolor}
\usepackage{textcomp}
\usepackage{alltt}
\usepackage{ifpdf}
\ifpdf
\usepackage[pdftex,
            pagebackref=true,
            colorlinks=true,
            linkcolor=blue,
            unicode
           ]{hyperref}
\else
\usepackage[ps2pdf,
            pagebackref=true,
            colorlinks=true,
            linkcolor=blue,
            unicode
           ]{hyperref}
\usepackage{pspicture}
\fi
\usepackage[utf8]{inputenc}
\usepackage{polski}
\usepackage[T1]{fontenc}

\usepackage{mathptmx}
\usepackage[scaled=.90]{helvet}
\usepackage{courier}
\usepackage{sectsty}
\usepackage{amssymb}
\usepackage[titles]{tocloft}
\usepackage{doxygen}
\lstset{language=C++,inputencoding=utf8,basicstyle=\footnotesize,breaklines=true,breakatwhitespace=true,tabsize=8,numbers=left }
\makeindex
\setcounter{tocdepth}{3}
\renewcommand{\footrulewidth}{0.4pt}
\renewcommand{\familydefault}{\sfdefault}
\hfuzz=15pt
\setlength{\emergencystretch}{15pt}
\hbadness=750
\tolerance=750
\begin{document}
\hypersetup{pageanchor=false,citecolor=blue}
\begin{titlepage}
\vspace*{7cm}
\begin{center}
{\Large P\-A\-M\-S\-Ilab7 }\\
\vspace*{1cm}
{\large Wygenerowano przez Doxygen 1.8.1.2}\\
\vspace*{0.5cm}
{\small Śr, 23 kwi 2014 22:06:03}\\
\end{center}
\end{titlepage}
\clearemptydoublepage
\pagenumbering{roman}
\tableofcontents
\clearemptydoublepage
\pagenumbering{arabic}
\hypersetup{pageanchor=true,citecolor=blue}
\chapter{Dokumentacja zadania P\-A\-M\-S\-I lab 7}
\label{index}\hypertarget{index}{}\begin{DoxyAuthor}{Autor}
Witold Zimnicki 
\end{DoxyAuthor}
\begin{DoxyDate}{Data}
19.\-4.\-2014 
\end{DoxyDate}

\chapter{Indeks klas}
\section{Lista klas}
Tutaj znajdują się klasy, struktury, unie i interfejsy wraz z ich krótkimi opisami\-:\begin{DoxyCompactList}
\item\contentsline{section}{\hyperlink{class_drzewo}{Drzewo} \\*Klasa \hyperlink{class_drzewo}{Drzewo} przedstawia drzewo binarne. Sklada sie z wielu wezlow }{\pageref{class_drzewo}}{}
\item\contentsline{section}{\hyperlink{class_hash_en}{Hash\-En} \\*Klasa \hyperlink{class_hash_en}{Hash\-En} przedstawia element mapy tablicy haszujacej }{\pageref{class_hash_en}}{}
\item\contentsline{section}{\hyperlink{class_hash_m}{Hash\-M} \\*Klasa \hyperlink{class_hash_m}{Hash\-M} przechowuje w sobie elementy (klucze i wartosci) w tablicy mieszajacej }{\pageref{class_hash_m}}{}
\item\contentsline{section}{\hyperlink{class_operacja}{Operacja} \\*Deklaracja klasy \hyperlink{class_operacja}{Operacja} }{\pageref{class_operacja}}{}
\item\contentsline{section}{\hyperlink{class_wezel}{Wezel} \\*Klasa \hyperlink{class_wezel}{Wezel} opisuje obiekt bedacy elementem wiekszej klasy \hyperlink{class_drzewo}{Drzewo} -\/ drzewa binarnego. Posiada on swoj klucz -\/ string, oraz wartosc -\/ int }{\pageref{class_wezel}}{}
\end{DoxyCompactList}

\chapter{Indeks plików}
\section{Lista plików}
Tutaj znajduje się lista wszystkich plików z ich krótkimi opisami\-:\begin{DoxyCompactList}
\item\contentsline{section}{/home/karolina/\-Pulpit/\-Project41\-Poker(1)/\-Project41\-Poker/\-Project41\-Poker/prj/inc/\hyperlink{_c_p_u_8h}{C\-P\-U.\-h} \\*Definicja klasy \hyperlink{class_c_p_u}{C\-P\-U} }{\pageref{_c_p_u_8h}}{}
\item\contentsline{section}{/home/karolina/\-Pulpit/\-Project41\-Poker(1)/\-Project41\-Poker/\-Project41\-Poker/prj/inc/\hyperlink{_g_r_a_8h}{G\-R\-A.\-h} \\*Definicja klasy \hyperlink{class_g_r_a}{G\-R\-A} }{\pageref{_g_r_a_8h}}{}
\item\contentsline{section}{/home/karolina/\-Pulpit/\-Project41\-Poker(1)/\-Project41\-Poker/\-Project41\-Poker/prj/inc/\hyperlink{gracz_8h}{gracz.\-h} \\*Definicja klasy \hyperlink{class_gracz}{Gracz} }{\pageref{gracz_8h}}{}
\item\contentsline{section}{/home/karolina/\-Pulpit/\-Project41\-Poker(1)/\-Project41\-Poker/\-Project41\-Poker/prj/inc/\hyperlink{interfejs_8h}{interfejs.\-h} \\*Definicja klasy \hyperlink{class_interfejs}{Interfejs} }{\pageref{interfejs_8h}}{}
\item\contentsline{section}{/home/karolina/\-Pulpit/\-Project41\-Poker(1)/\-Project41\-Poker/\-Project41\-Poker/prj/inc/\hyperlink{karta_8h}{karta.\-h} \\*Definicja klasy \hyperlink{class_karta}{Karta} }{\pageref{karta_8h}}{}
\item\contentsline{section}{/home/karolina/\-Pulpit/\-Project41\-Poker(1)/\-Project41\-Poker/\-Project41\-Poker/prj/inc/\hyperlink{talia_8h}{talia.\-h} \\*Definicja klasy \hyperlink{class_talia}{Talia} }{\pageref{talia_8h}}{}
\item\contentsline{section}{/home/karolina/\-Pulpit/\-Project41\-Poker(1)/\-Project41\-Poker/\-Project41\-Poker/prj/inc/\hyperlink{zestaw_8h}{zestaw.\-h} \\*Definicja klasy \hyperlink{class_zestaw}{Zestaw} }{\pageref{zestaw_8h}}{}
\item\contentsline{section}{/home/karolina/\-Pulpit/\-Project41\-Poker(1)/\-Project41\-Poker/\-Project41\-Poker/prj/src/\hyperlink{_c_p_u_8cpp}{C\-P\-U.\-cpp} \\*Definicja metody klasy \hyperlink{class_c_p_u}{C\-P\-U} }{\pageref{_c_p_u_8cpp}}{}
\item\contentsline{section}{/home/karolina/\-Pulpit/\-Project41\-Poker(1)/\-Project41\-Poker/\-Project41\-Poker/prj/src/\hyperlink{_g_r_a_8cpp}{G\-R\-A.\-cpp} \\*Definicja metody klasy \hyperlink{class_g_r_a}{G\-R\-A} }{\pageref{_g_r_a_8cpp}}{}
\item\contentsline{section}{/home/karolina/\-Pulpit/\-Project41\-Poker(1)/\-Project41\-Poker/\-Project41\-Poker/prj/src/\hyperlink{gracz_8cpp}{gracz.\-cpp} \\*Definicja metody klasy \hyperlink{class_gracz}{Gracz} }{\pageref{gracz_8cpp}}{}
\item\contentsline{section}{/home/karolina/\-Pulpit/\-Project41\-Poker(1)/\-Project41\-Poker/\-Project41\-Poker/prj/src/\hyperlink{interfejs_8cpp}{interfejs.\-cpp} \\*Definicja metody klasy \hyperlink{class_interfejs}{Interfejs} }{\pageref{interfejs_8cpp}}{}
\item\contentsline{section}{/home/karolina/\-Pulpit/\-Project41\-Poker(1)/\-Project41\-Poker/\-Project41\-Poker/prj/src/\hyperlink{karta_8cpp}{karta.\-cpp} \\*Definicja metody klasy \hyperlink{class_karta}{Karta} }{\pageref{karta_8cpp}}{}
\item\contentsline{section}{/home/karolina/\-Pulpit/\-Project41\-Poker(1)/\-Project41\-Poker/\-Project41\-Poker/prj/src/\hyperlink{main_8cpp}{main.\-cpp} \\*Plik glowny programu }{\pageref{main_8cpp}}{}
\item\contentsline{section}{/home/karolina/\-Pulpit/\-Project41\-Poker(1)/\-Project41\-Poker/\-Project41\-Poker/prj/src/\hyperlink{talia_8cpp}{talia.\-cpp} \\*Definicja metody klasy \hyperlink{class_talia}{Talia} }{\pageref{talia_8cpp}}{}
\item\contentsline{section}{/home/karolina/\-Pulpit/\-Project41\-Poker(1)/\-Project41\-Poker/\-Project41\-Poker/prj/src/\hyperlink{zestaw_8cpp}{zestaw.\-cpp} \\*Definicja metody klasy \hyperlink{class_zestaw}{Zestaw} }{\pageref{zestaw_8cpp}}{}
\end{DoxyCompactList}

\chapter{Dokumentacja klas}
\hypertarget{class_drzewo}{\section{Dokumentacja klasy Drzewo}
\label{class_drzewo}\index{Drzewo@{Drzewo}}
}


Klasa \hyperlink{class_drzewo}{Drzewo} przedstawia drzewo binarne. Sklada sie z wielu wezlow.  




{\ttfamily \#include $<$drzewo.\-h$>$}

\subsection*{Metody publiczne}
\begin{DoxyCompactItemize}
\item 
int \hyperlink{class_drzewo_a3cc4b6778603a1bb0d98c4435d5cf068}{Wyszukaj\-Wartosc} (string kluczyk)
\begin{DoxyCompactList}\small\item\em Funkcja wyszukuje wartosc po kluczu. \end{DoxyCompactList}\item 
\hyperlink{class_drzewo_a309a5197b6a02c761f046fc99136eebb}{Drzewo} ()
\begin{DoxyCompactList}\small\item\em Konstruktor domyslny, zeruje korzen. \end{DoxyCompactList}\item 
bool \hyperlink{class_drzewo_a50373a0a15637acc4e8f961b1a53cb37}{dodaj} (int wartosc, string kluczyk)
\begin{DoxyCompactList}\small\item\em Funkcja dodaje wybrana pare \-: klucz oraz wartosc. \end{DoxyCompactList}\item 
void \hyperlink{class_drzewo_a3d58680a7f5d118119ca564c6a4b1018}{Wyswietlpokolei} ()
\begin{DoxyCompactList}\small\item\em Funkcja wyswietlajaca elementy po kolei. \end{DoxyCompactList}\end{DoxyCompactItemize}


\subsection{Opis szczegółowy}
Klasa \hyperlink{class_drzewo}{Drzewo} przedstawia drzewo binarne. Sklada sie z wielu wezlow. 

\subsection{Dokumentacja konstruktora i destruktora}
\hypertarget{class_drzewo_a309a5197b6a02c761f046fc99136eebb}{\index{Drzewo@{Drzewo}!Drzewo@{Drzewo}}
\index{Drzewo@{Drzewo}!Drzewo@{Drzewo}}
\subsubsection[{Drzewo}]{\setlength{\rightskip}{0pt plus 5cm}Drzewo\-::\-Drzewo (
\begin{DoxyParamCaption}
{}
\end{DoxyParamCaption}
)\hspace{0.3cm}{\ttfamily [inline]}}}\label{class_drzewo_a309a5197b6a02c761f046fc99136eebb}


Konstruktor domyslny, zeruje korzen. 



\subsection{Dokumentacja funkcji składowych}
\hypertarget{class_drzewo_a50373a0a15637acc4e8f961b1a53cb37}{\index{Drzewo@{Drzewo}!dodaj@{dodaj}}
\index{dodaj@{dodaj}!Drzewo@{Drzewo}}
\subsubsection[{dodaj}]{\setlength{\rightskip}{0pt plus 5cm}bool Drzewo\-::dodaj (
\begin{DoxyParamCaption}
\item[{int}]{wartosc, }
\item[{string}]{kluczyk}
\end{DoxyParamCaption}
)}}\label{class_drzewo_a50373a0a15637acc4e8f961b1a53cb37}


Funkcja dodaje wybrana pare \-: klucz oraz wartosc. 

\begin{DoxyVerb}    Funkcja dodajaca element i klucz dziala nastepujaco: jezeli drzewo jest puste, to tworzy nowy wezel bedacy poczatkiem tego drzewa. Jezeli drzewo istnieje, po kolei sprawdzajac z lewej i prawej strony odpowiednie wartosci
    tworza sie nowe wezly.
\end{DoxyVerb}
 Najwazniejsze pola i argumenty funkcji\-: -\/tmp -\/ wskaznik pomocniczny wskazujacy na korzen

\begin{DoxyReturn}{Zwraca}
true, jesli dodana zostala nowa wartosc; false, jesli dana para juz istnieje 
\end{DoxyReturn}
\hypertarget{class_drzewo_a3d58680a7f5d118119ca564c6a4b1018}{\index{Drzewo@{Drzewo}!Wyswietlpokolei@{Wyswietlpokolei}}
\index{Wyswietlpokolei@{Wyswietlpokolei}!Drzewo@{Drzewo}}
\subsubsection[{Wyswietlpokolei}]{\setlength{\rightskip}{0pt plus 5cm}void Drzewo\-::\-Wyswietlpokolei (
\begin{DoxyParamCaption}
{}
\end{DoxyParamCaption}
)\hspace{0.3cm}{\ttfamily [inline]}}}\label{class_drzewo_a3d58680a7f5d118119ca564c6a4b1018}


Funkcja wyswietlajaca elementy po kolei. 

\hypertarget{class_drzewo_a3cc4b6778603a1bb0d98c4435d5cf068}{\index{Drzewo@{Drzewo}!Wyszukaj\-Wartosc@{Wyszukaj\-Wartosc}}
\index{Wyszukaj\-Wartosc@{Wyszukaj\-Wartosc}!Drzewo@{Drzewo}}
\subsubsection[{Wyszukaj\-Wartosc}]{\setlength{\rightskip}{0pt plus 5cm}int Drzewo\-::\-Wyszukaj\-Wartosc (
\begin{DoxyParamCaption}
\item[{string}]{kluczyk}
\end{DoxyParamCaption}
)}}\label{class_drzewo_a3cc4b6778603a1bb0d98c4435d5cf068}


Funkcja wyszukuje wartosc po kluczu. 

\begin{DoxyReturn}{Zwraca}
wartosc szukanego klucza. 
\end{DoxyReturn}


Dokumentacja dla tej klasy została wygenerowana z plików\-:\begin{DoxyCompactItemize}
\item 
C\-:/\-Users/\-Witek/\-Documents/\-Visual Studio 2012/\-Projects/\-Project47\-P\-A\-M\-S\-Ilab7/\-Project47\-P\-A\-M\-S\-Ilab7/\hyperlink{drzewo_8h}{drzewo.\-h}\item 
C\-:/\-Users/\-Witek/\-Documents/\-Visual Studio 2012/\-Projects/\-Project47\-P\-A\-M\-S\-Ilab7/\-Project47\-P\-A\-M\-S\-Ilab7/\hyperlink{drzewo_8cpp}{drzewo.\-cpp}\end{DoxyCompactItemize}

\hypertarget{class_hash_en}{\section{Dokumentacja klasy Hash\-En}
\label{class_hash_en}\index{Hash\-En@{Hash\-En}}
}


Klasa \hyperlink{class_hash_en}{Hash\-En} przedstawia element mapy tablicy haszujacej.  




{\ttfamily \#include $<$hashen.\-h$>$}

\subsection*{Metody publiczne}
\begin{DoxyCompactItemize}
\item 
\hyperlink{class_hash_en_ac38e1fb08850316a1afca8f7f2994d8e}{Hash\-En} (string klucz, int wartosc)
\item 
string \hyperlink{class_hash_en_a6f692181701d5cc999808e8c40c1ad3a}{Pobierz\-Klucz} ()
\item 
int \hyperlink{class_hash_en_a1907b6a4315ed056c0d94cb086917403}{Pobierz\-Wartosc} ()
\end{DoxyCompactItemize}


\subsection{Opis szczegółowy}
Klasa \hyperlink{class_hash_en}{Hash\-En} przedstawia element mapy tablicy haszujacej. 

\subsection{Dokumentacja konstruktora i destruktora}
\hypertarget{class_hash_en_ac38e1fb08850316a1afca8f7f2994d8e}{\index{Hash\-En@{Hash\-En}!Hash\-En@{Hash\-En}}
\index{Hash\-En@{Hash\-En}!HashEn@{Hash\-En}}
\subsubsection[{Hash\-En}]{\setlength{\rightskip}{0pt plus 5cm}Hash\-En\-::\-Hash\-En (
\begin{DoxyParamCaption}
\item[{string}]{klucz, }
\item[{int}]{wartosc}
\end{DoxyParamCaption}
)}}\label{class_hash_en_ac38e1fb08850316a1afca8f7f2994d8e}


\subsection{Dokumentacja funkcji składowych}
\hypertarget{class_hash_en_a6f692181701d5cc999808e8c40c1ad3a}{\index{Hash\-En@{Hash\-En}!Pobierz\-Klucz@{Pobierz\-Klucz}}
\index{Pobierz\-Klucz@{Pobierz\-Klucz}!HashEn@{Hash\-En}}
\subsubsection[{Pobierz\-Klucz}]{\setlength{\rightskip}{0pt plus 5cm}string Hash\-En\-::\-Pobierz\-Klucz (
\begin{DoxyParamCaption}
{}
\end{DoxyParamCaption}
)}}\label{class_hash_en_a6f692181701d5cc999808e8c40c1ad3a}
\hypertarget{class_hash_en_a1907b6a4315ed056c0d94cb086917403}{\index{Hash\-En@{Hash\-En}!Pobierz\-Wartosc@{Pobierz\-Wartosc}}
\index{Pobierz\-Wartosc@{Pobierz\-Wartosc}!HashEn@{Hash\-En}}
\subsubsection[{Pobierz\-Wartosc}]{\setlength{\rightskip}{0pt plus 5cm}int Hash\-En\-::\-Pobierz\-Wartosc (
\begin{DoxyParamCaption}
{}
\end{DoxyParamCaption}
)}}\label{class_hash_en_a1907b6a4315ed056c0d94cb086917403}


Dokumentacja dla tej klasy została wygenerowana z plików\-:\begin{DoxyCompactItemize}
\item 
C\-:/\-Users/\-Witek/\-Documents/\-Visual Studio 2012/\-Projects/\-Project47\-P\-A\-M\-S\-Ilab7/\-Project47\-P\-A\-M\-S\-Ilab7/\hyperlink{hashen_8h}{hashen.\-h}\item 
C\-:/\-Users/\-Witek/\-Documents/\-Visual Studio 2012/\-Projects/\-Project47\-P\-A\-M\-S\-Ilab7/\-Project47\-P\-A\-M\-S\-Ilab7/\hyperlink{hashen_8cpp}{hashen.\-cpp}\end{DoxyCompactItemize}

\hypertarget{class_hash_m}{\section{Dokumentacja klasy Hash\-M}
\label{class_hash_m}\index{Hash\-M@{Hash\-M}}
}


Klasa \hyperlink{class_hash_m}{Hash\-M} przechowuje w sobie elementy (klucze i wartosci) w tablicy mieszajacej.  




{\ttfamily \#include $<$hashm.\-h$>$}

\subsection*{Metody publiczne}
\begin{DoxyCompactItemize}
\item 
\hyperlink{class_hash_m_a7fcc78711f22135e8f485bcce98bf474}{Hash\-M} (int r)
\begin{DoxyCompactList}\small\item\em Konstruktor parametryczny tworzacy obiekt Hasz\-M o danym rozmiarze. \end{DoxyCompactList}\item 
int \hyperlink{class_hash_m_a58cc6b9f3a555c112e21d89b5ff1155c}{Znajdz} (string klucz)
\begin{DoxyCompactList}\small\item\em Funkcja znachodzi i zwraca wartosc po kluczu. \end{DoxyCompactList}\item 
void \hyperlink{class_hash_m_a522c506c39f2328f347ce9b73a69a40c}{Push} (string key, int value)
\begin{DoxyCompactList}\small\item\em Funkcja dodajaca pare klucza oraz wartosci do tablicy. \end{DoxyCompactList}\item 
void \hyperlink{class_hash_m_a07d27c9280d89fc04708d5b60c56b42c}{wypisz} ()
\begin{DoxyCompactList}\small\item\em Funkcja wypisuje elementy tablicy haszujacej. \end{DoxyCompactList}\item 
\hyperlink{class_hash_m_a1d67f4833851676fa84cb7d6eb02ae19}{Hash\-M} ()
\begin{DoxyCompactList}\small\item\em Konstruktor domy�lny. \end{DoxyCompactList}\end{DoxyCompactItemize}
\subsection*{Atrybuty publiczne}
\begin{DoxyCompactItemize}
\item 
int \hyperlink{class_hash_m_aa13e6f1551c1f03b65df26d87d7ae6f8}{rozmiar}
\begin{DoxyCompactList}\small\item\em pole przechowuje rozmiar tablicy. \end{DoxyCompactList}\end{DoxyCompactItemize}


\subsection{Opis szczegółowy}
Klasa \hyperlink{class_hash_m}{Hash\-M} przechowuje w sobie elementy (klucze i wartosci) w tablicy mieszajacej. 

\subsection{Dokumentacja konstruktora i destruktora}
\hypertarget{class_hash_m_a7fcc78711f22135e8f485bcce98bf474}{\index{Hash\-M@{Hash\-M}!Hash\-M@{Hash\-M}}
\index{Hash\-M@{Hash\-M}!HashM@{Hash\-M}}
\subsubsection[{Hash\-M}]{\setlength{\rightskip}{0pt plus 5cm}Hash\-M\-::\-Hash\-M (
\begin{DoxyParamCaption}
\item[{int}]{r}
\end{DoxyParamCaption}
)}}\label{class_hash_m_a7fcc78711f22135e8f485bcce98bf474}


Konstruktor parametryczny tworzacy obiekt Hasz\-M o danym rozmiarze. 

Konstruktor alokuje pamiec o zadanej ilosci dla tablicy haszujacej. \hypertarget{class_hash_m_a1d67f4833851676fa84cb7d6eb02ae19}{\index{Hash\-M@{Hash\-M}!Hash\-M@{Hash\-M}}
\index{Hash\-M@{Hash\-M}!HashM@{Hash\-M}}
\subsubsection[{Hash\-M}]{\setlength{\rightskip}{0pt plus 5cm}Hash\-M\-::\-Hash\-M (
\begin{DoxyParamCaption}
{}
\end{DoxyParamCaption}
)\hspace{0.3cm}{\ttfamily [inline]}}}\label{class_hash_m_a1d67f4833851676fa84cb7d6eb02ae19}


Konstruktor domy�lny. 



\subsection{Dokumentacja funkcji składowych}
\hypertarget{class_hash_m_a522c506c39f2328f347ce9b73a69a40c}{\index{Hash\-M@{Hash\-M}!Push@{Push}}
\index{Push@{Push}!HashM@{Hash\-M}}
\subsubsection[{Push}]{\setlength{\rightskip}{0pt plus 5cm}void Hash\-M\-::\-Push (
\begin{DoxyParamCaption}
\item[{string}]{key, }
\item[{int}]{value}
\end{DoxyParamCaption}
)}}\label{class_hash_m_a522c506c39f2328f347ce9b73a69a40c}


Funkcja dodajaca pare klucza oraz wartosci do tablicy. 

\hypertarget{class_hash_m_a07d27c9280d89fc04708d5b60c56b42c}{\index{Hash\-M@{Hash\-M}!wypisz@{wypisz}}
\index{wypisz@{wypisz}!HashM@{Hash\-M}}
\subsubsection[{wypisz}]{\setlength{\rightskip}{0pt plus 5cm}void Hash\-M\-::wypisz (
\begin{DoxyParamCaption}
{}
\end{DoxyParamCaption}
)}}\label{class_hash_m_a07d27c9280d89fc04708d5b60c56b42c}


Funkcja wypisuje elementy tablicy haszujacej. 

\hypertarget{class_hash_m_a58cc6b9f3a555c112e21d89b5ff1155c}{\index{Hash\-M@{Hash\-M}!Znajdz@{Znajdz}}
\index{Znajdz@{Znajdz}!HashM@{Hash\-M}}
\subsubsection[{Znajdz}]{\setlength{\rightskip}{0pt plus 5cm}int Hash\-M\-::\-Znajdz (
\begin{DoxyParamCaption}
\item[{string}]{klucz}
\end{DoxyParamCaption}
)}}\label{class_hash_m_a58cc6b9f3a555c112e21d89b5ff1155c}


Funkcja znachodzi i zwraca wartosc po kluczu. 

\begin{DoxyReturn}{Zwraca}
Wartosc dla wybranego klucza, lub -\/1, jesli tablica jest pusta. 
\end{DoxyReturn}


\subsection{Dokumentacja atrybutów składowych}
\hypertarget{class_hash_m_aa13e6f1551c1f03b65df26d87d7ae6f8}{\index{Hash\-M@{Hash\-M}!rozmiar@{rozmiar}}
\index{rozmiar@{rozmiar}!HashM@{Hash\-M}}
\subsubsection[{rozmiar}]{\setlength{\rightskip}{0pt plus 5cm}int Hash\-M\-::rozmiar}}\label{class_hash_m_aa13e6f1551c1f03b65df26d87d7ae6f8}


pole przechowuje rozmiar tablicy. 



Dokumentacja dla tej klasy została wygenerowana z plików\-:\begin{DoxyCompactItemize}
\item 
C\-:/\-Users/\-Witek/\-Documents/\-Visual Studio 2012/\-Projects/\-Project47\-P\-A\-M\-S\-Ilab7/\-Project47\-P\-A\-M\-S\-Ilab7/\hyperlink{hashm_8h}{hashm.\-h}\item 
C\-:/\-Users/\-Witek/\-Documents/\-Visual Studio 2012/\-Projects/\-Project47\-P\-A\-M\-S\-Ilab7/\-Project47\-P\-A\-M\-S\-Ilab7/\hyperlink{hashm_8cpp}{hashm.\-cpp}\end{DoxyCompactItemize}

\hypertarget{class_operacja}{\section{Dokumentacja klasy Operacja}
\label{class_operacja}\index{Operacja@{Operacja}}
}


Deklaracja klasy \hyperlink{class_operacja}{Operacja}.  




{\ttfamily \#include $<$operacja.\-h$>$}

\subsection*{Metody publiczne}
\begin{DoxyCompactItemize}
\item 
void \hyperlink{class_operacja_a8a31476894d307c400f965dd3dfcbb46}{Zmierz\-Czas\-Start} ()
\begin{DoxyCompactList}\small\item\em Funkcja wyznaczajaca poczatek pomiaru czasu. \end{DoxyCompactList}\item 
void \hyperlink{class_operacja_a971e3493bec71fc140ac44f74bd92998}{Zmierz\-Czas\-Koniec\-Drzewo} ()
\begin{DoxyCompactList}\small\item\em Funkcja wyznaczajaca koniec pomiaru czasu dla zapelniania drzewa binarnego. \end{DoxyCompactList}\item 
void \hyperlink{class_operacja_a39ea8e62797c94693c75abe6f0416035}{Zmierz\-Czas\-Koniec\-Hasz} ()
\begin{DoxyCompactList}\small\item\em Funkcja wyznaczajaca koniec pomiaru czasu dla zapelniania tablicy haszujacej. \end{DoxyCompactList}\item 
void \hyperlink{class_operacja_a3f6cb38d924c260711fad70fe0eed35f}{Pobierz\-Ilosc\-Powtorzen} ()
\begin{DoxyCompactList}\small\item\em Funkcja pobierajaca ilosc powtorzen od uzytkownika. \end{DoxyCompactList}\item 
void \hyperlink{class_operacja_aca61c478012e3477b123b1cccedd375c}{Policz\-Operacje\-Drzewo} ()
\begin{DoxyCompactList}\small\item\em Funkcja wykonujaca zapelnianie drzewa binarnego. \end{DoxyCompactList}\item 
void \hyperlink{class_operacja_a97f0315ebc8d718a497415023ef11564}{Policz\-Operacje\-Hasz} ()
\begin{DoxyCompactList}\small\item\em Funkcja wykonujaca zapelnianie tablicy haszujacej. \end{DoxyCompactList}\item 
void \hyperlink{class_operacja_add48398ebfeae90a0d1b58a31962fd7d}{Dzialaj} ()
\begin{DoxyCompactList}\small\item\em Glowna Funkcja Operacji wykonujaca wszystkie potrzebne funkcje do uzyskania czasow. \end{DoxyCompactList}\item 
\hyperlink{class_operacja_a1624fb5817c0b60e1680509fc4517732}{Operacja} ()
\begin{DoxyCompactList}\small\item\em Konstruktor bezparametryczny. \end{DoxyCompactList}\end{DoxyCompactItemize}
\subsection*{Atrybuty publiczne}
\begin{DoxyCompactItemize}
\item 
\hyperlink{class_hash_m}{Hash\-M} \hyperlink{class_operacja_ad67a2aa20b71ec0e1f1cff12b7220b6a}{haszujaca}
\begin{DoxyCompactList}\small\item\em Tablica haszujaca dla danej operacji. \end{DoxyCompactList}\item 
\hyperlink{class_drzewo}{Drzewo} \hyperlink{class_operacja_a814e223e988958ccbbe66aef89054da3}{drzewko}
\begin{DoxyCompactList}\small\item\em \hyperlink{class_drzewo}{Drzewo} binarne dla danej operacji. \end{DoxyCompactList}\item 
int \hyperlink{class_operacja_aa568b17d05f31132b3d97eb5b7e93d61}{Powtorzenia}
\begin{DoxyCompactList}\small\item\em Pole przechowujace informiacji o ilosci powtorzen dla danej operacji. \end{DoxyCompactList}\end{DoxyCompactItemize}


\subsection{Opis szczegółowy}
Deklaracja klasy \hyperlink{class_operacja}{Operacja}. 

Klasa \hyperlink{class_operacja}{Operacja} posiada pola oraz funkcje potrzebne do wykonywania dzialan na roznych strukturach danych. 

\subsection{Dokumentacja konstruktora i destruktora}
\hypertarget{class_operacja_a1624fb5817c0b60e1680509fc4517732}{\index{Operacja@{Operacja}!Operacja@{Operacja}}
\index{Operacja@{Operacja}!Operacja@{Operacja}}
\subsubsection[{Operacja}]{\setlength{\rightskip}{0pt plus 5cm}Operacja\-::\-Operacja (
\begin{DoxyParamCaption}
{}
\end{DoxyParamCaption}
)}}\label{class_operacja_a1624fb5817c0b60e1680509fc4517732}


Konstruktor bezparametryczny. 

Konstruktor pobiera wybrana ilosc zestawow, ktore chcemy dodac do tablicy i drzewa. 

\subsection{Dokumentacja funkcji składowych}
\hypertarget{class_operacja_add48398ebfeae90a0d1b58a31962fd7d}{\index{Operacja@{Operacja}!Dzialaj@{Dzialaj}}
\index{Dzialaj@{Dzialaj}!Operacja@{Operacja}}
\subsubsection[{Dzialaj}]{\setlength{\rightskip}{0pt plus 5cm}void Operacja\-::\-Dzialaj (
\begin{DoxyParamCaption}
{}
\end{DoxyParamCaption}
)}}\label{class_operacja_add48398ebfeae90a0d1b58a31962fd7d}


Glowna Funkcja Operacji wykonujaca wszystkie potrzebne funkcje do uzyskania czasow. 

\hypertarget{class_operacja_a3f6cb38d924c260711fad70fe0eed35f}{\index{Operacja@{Operacja}!Pobierz\-Ilosc\-Powtorzen@{Pobierz\-Ilosc\-Powtorzen}}
\index{Pobierz\-Ilosc\-Powtorzen@{Pobierz\-Ilosc\-Powtorzen}!Operacja@{Operacja}}
\subsubsection[{Pobierz\-Ilosc\-Powtorzen}]{\setlength{\rightskip}{0pt plus 5cm}void Operacja\-::\-Pobierz\-Ilosc\-Powtorzen (
\begin{DoxyParamCaption}
{}
\end{DoxyParamCaption}
)}}\label{class_operacja_a3f6cb38d924c260711fad70fe0eed35f}


Funkcja pobierajaca ilosc powtorzen od uzytkownika. 

\hypertarget{class_operacja_aca61c478012e3477b123b1cccedd375c}{\index{Operacja@{Operacja}!Policz\-Operacje\-Drzewo@{Policz\-Operacje\-Drzewo}}
\index{Policz\-Operacje\-Drzewo@{Policz\-Operacje\-Drzewo}!Operacja@{Operacja}}
\subsubsection[{Policz\-Operacje\-Drzewo}]{\setlength{\rightskip}{0pt plus 5cm}void Operacja\-::\-Policz\-Operacje\-Drzewo (
\begin{DoxyParamCaption}
{}
\end{DoxyParamCaption}
)}}\label{class_operacja_aca61c478012e3477b123b1cccedd375c}


Funkcja wykonujaca zapelnianie drzewa binarnego. 

\hypertarget{class_operacja_a97f0315ebc8d718a497415023ef11564}{\index{Operacja@{Operacja}!Policz\-Operacje\-Hasz@{Policz\-Operacje\-Hasz}}
\index{Policz\-Operacje\-Hasz@{Policz\-Operacje\-Hasz}!Operacja@{Operacja}}
\subsubsection[{Policz\-Operacje\-Hasz}]{\setlength{\rightskip}{0pt plus 5cm}void Operacja\-::\-Policz\-Operacje\-Hasz (
\begin{DoxyParamCaption}
{}
\end{DoxyParamCaption}
)}}\label{class_operacja_a97f0315ebc8d718a497415023ef11564}


Funkcja wykonujaca zapelnianie tablicy haszujacej. 

\hypertarget{class_operacja_a971e3493bec71fc140ac44f74bd92998}{\index{Operacja@{Operacja}!Zmierz\-Czas\-Koniec\-Drzewo@{Zmierz\-Czas\-Koniec\-Drzewo}}
\index{Zmierz\-Czas\-Koniec\-Drzewo@{Zmierz\-Czas\-Koniec\-Drzewo}!Operacja@{Operacja}}
\subsubsection[{Zmierz\-Czas\-Koniec\-Drzewo}]{\setlength{\rightskip}{0pt plus 5cm}void Operacja\-::\-Zmierz\-Czas\-Koniec\-Drzewo (
\begin{DoxyParamCaption}
{}
\end{DoxyParamCaption}
)}}\label{class_operacja_a971e3493bec71fc140ac44f74bd92998}


Funkcja wyznaczajaca koniec pomiaru czasu dla zapelniania drzewa binarnego. 

\hypertarget{class_operacja_a39ea8e62797c94693c75abe6f0416035}{\index{Operacja@{Operacja}!Zmierz\-Czas\-Koniec\-Hasz@{Zmierz\-Czas\-Koniec\-Hasz}}
\index{Zmierz\-Czas\-Koniec\-Hasz@{Zmierz\-Czas\-Koniec\-Hasz}!Operacja@{Operacja}}
\subsubsection[{Zmierz\-Czas\-Koniec\-Hasz}]{\setlength{\rightskip}{0pt plus 5cm}void Operacja\-::\-Zmierz\-Czas\-Koniec\-Hasz (
\begin{DoxyParamCaption}
{}
\end{DoxyParamCaption}
)}}\label{class_operacja_a39ea8e62797c94693c75abe6f0416035}


Funkcja wyznaczajaca koniec pomiaru czasu dla zapelniania tablicy haszujacej. 

\hypertarget{class_operacja_a8a31476894d307c400f965dd3dfcbb46}{\index{Operacja@{Operacja}!Zmierz\-Czas\-Start@{Zmierz\-Czas\-Start}}
\index{Zmierz\-Czas\-Start@{Zmierz\-Czas\-Start}!Operacja@{Operacja}}
\subsubsection[{Zmierz\-Czas\-Start}]{\setlength{\rightskip}{0pt plus 5cm}void Operacja\-::\-Zmierz\-Czas\-Start (
\begin{DoxyParamCaption}
{}
\end{DoxyParamCaption}
)}}\label{class_operacja_a8a31476894d307c400f965dd3dfcbb46}


Funkcja wyznaczajaca poczatek pomiaru czasu. 



\subsection{Dokumentacja atrybutów składowych}
\hypertarget{class_operacja_a814e223e988958ccbbe66aef89054da3}{\index{Operacja@{Operacja}!drzewko@{drzewko}}
\index{drzewko@{drzewko}!Operacja@{Operacja}}
\subsubsection[{drzewko}]{\setlength{\rightskip}{0pt plus 5cm}{\bf Drzewo} Operacja\-::drzewko}}\label{class_operacja_a814e223e988958ccbbe66aef89054da3}


\hyperlink{class_drzewo}{Drzewo} binarne dla danej operacji. 

\hypertarget{class_operacja_ad67a2aa20b71ec0e1f1cff12b7220b6a}{\index{Operacja@{Operacja}!haszujaca@{haszujaca}}
\index{haszujaca@{haszujaca}!Operacja@{Operacja}}
\subsubsection[{haszujaca}]{\setlength{\rightskip}{0pt plus 5cm}{\bf Hash\-M} Operacja\-::haszujaca}}\label{class_operacja_ad67a2aa20b71ec0e1f1cff12b7220b6a}


Tablica haszujaca dla danej operacji. 

\hypertarget{class_operacja_aa568b17d05f31132b3d97eb5b7e93d61}{\index{Operacja@{Operacja}!Powtorzenia@{Powtorzenia}}
\index{Powtorzenia@{Powtorzenia}!Operacja@{Operacja}}
\subsubsection[{Powtorzenia}]{\setlength{\rightskip}{0pt plus 5cm}int Operacja\-::\-Powtorzenia}}\label{class_operacja_aa568b17d05f31132b3d97eb5b7e93d61}


Pole przechowujace informiacji o ilosci powtorzen dla danej operacji. 



Dokumentacja dla tej klasy została wygenerowana z plików\-:\begin{DoxyCompactItemize}
\item 
C\-:/\-Users/\-Witek/\-Documents/\-Visual Studio 2012/\-Projects/\-Project47\-P\-A\-M\-S\-Ilab7/\-Project47\-P\-A\-M\-S\-Ilab7/\hyperlink{operacja_8h}{operacja.\-h}\item 
C\-:/\-Users/\-Witek/\-Documents/\-Visual Studio 2012/\-Projects/\-Project47\-P\-A\-M\-S\-Ilab7/\-Project47\-P\-A\-M\-S\-Ilab7/\hyperlink{operacja_8cpp}{operacja.\-cpp}\end{DoxyCompactItemize}

\hypertarget{class_wezel}{\section{Dokumentacja klasy Wezel}
\label{class_wezel}\index{Wezel@{Wezel}}
}


Klasa \hyperlink{class_wezel}{Wezel} opisuje obiekt bedacy elementem wiekszej klasy \hyperlink{class_drzewo}{Drzewo} -\/ drzewa binarnego. Posiada on swoj klucz -\/ string, oraz wartosc -\/ int.  




{\ttfamily \#include $<$wezel.\-h$>$}

\subsection*{Metody publiczne}
\begin{DoxyCompactItemize}
\item 
\hyperlink{class_wezel_a4b0f983c449278e4285687fdcddb5415}{Wezel} ()
\begin{DoxyCompactList}\small\item\em konstruktor domyslny -\/ zeruje pola. \end{DoxyCompactList}\end{DoxyCompactItemize}
\subsection*{Atrybuty publiczne}
\begin{DoxyCompactItemize}
\item 
string \hyperlink{class_wezel_a69170fb72d1770a86e0dee946536cb9f}{klucz}
\begin{DoxyCompactList}\small\item\em klucz wezla. \end{DoxyCompactList}\item 
int \hyperlink{class_wezel_ab26d2571cea51b8e4494fd8c39126416}{wartosc}
\begin{DoxyCompactList}\small\item\em wartosc wezla. \end{DoxyCompactList}\item 
\hyperlink{class_wezel}{Wezel} $\ast$ \hyperlink{class_wezel_a4a0cad770f3dedf11c3f656a832f7d3e}{lewy}
\begin{DoxyCompactList}\small\item\em wska�nik na lewego potomka. \end{DoxyCompactList}\item 
\hyperlink{class_wezel}{Wezel} $\ast$ \hyperlink{class_wezel_aa361ee6a1f625e258bad53c94420590a}{prawy}
\begin{DoxyCompactList}\small\item\em wska�nik na prawego potomka. \end{DoxyCompactList}\end{DoxyCompactItemize}


\subsection{Opis szczegółowy}
Klasa \hyperlink{class_wezel}{Wezel} opisuje obiekt bedacy elementem wiekszej klasy \hyperlink{class_drzewo}{Drzewo} -\/ drzewa binarnego. Posiada on swoj klucz -\/ string, oraz wartosc -\/ int. 

\subsection{Dokumentacja konstruktora i destruktora}
\hypertarget{class_wezel_a4b0f983c449278e4285687fdcddb5415}{\index{Wezel@{Wezel}!Wezel@{Wezel}}
\index{Wezel@{Wezel}!Wezel@{Wezel}}
\subsubsection[{Wezel}]{\setlength{\rightskip}{0pt plus 5cm}Wezel\-::\-Wezel (
\begin{DoxyParamCaption}
{}
\end{DoxyParamCaption}
)\hspace{0.3cm}{\ttfamily [inline]}}}\label{class_wezel_a4b0f983c449278e4285687fdcddb5415}


konstruktor domyslny -\/ zeruje pola. 



\subsection{Dokumentacja atrybutów składowych}
\hypertarget{class_wezel_a69170fb72d1770a86e0dee946536cb9f}{\index{Wezel@{Wezel}!klucz@{klucz}}
\index{klucz@{klucz}!Wezel@{Wezel}}
\subsubsection[{klucz}]{\setlength{\rightskip}{0pt plus 5cm}string Wezel\-::klucz}}\label{class_wezel_a69170fb72d1770a86e0dee946536cb9f}


klucz wezla. 

\hypertarget{class_wezel_a4a0cad770f3dedf11c3f656a832f7d3e}{\index{Wezel@{Wezel}!lewy@{lewy}}
\index{lewy@{lewy}!Wezel@{Wezel}}
\subsubsection[{lewy}]{\setlength{\rightskip}{0pt plus 5cm}{\bf Wezel}$\ast$ Wezel\-::lewy}}\label{class_wezel_a4a0cad770f3dedf11c3f656a832f7d3e}


wska�nik na lewego potomka. 

\hypertarget{class_wezel_aa361ee6a1f625e258bad53c94420590a}{\index{Wezel@{Wezel}!prawy@{prawy}}
\index{prawy@{prawy}!Wezel@{Wezel}}
\subsubsection[{prawy}]{\setlength{\rightskip}{0pt plus 5cm}{\bf Wezel}$\ast$ Wezel\-::prawy}}\label{class_wezel_aa361ee6a1f625e258bad53c94420590a}


wska�nik na prawego potomka. 

\hypertarget{class_wezel_ab26d2571cea51b8e4494fd8c39126416}{\index{Wezel@{Wezel}!wartosc@{wartosc}}
\index{wartosc@{wartosc}!Wezel@{Wezel}}
\subsubsection[{wartosc}]{\setlength{\rightskip}{0pt plus 5cm}int Wezel\-::wartosc}}\label{class_wezel_ab26d2571cea51b8e4494fd8c39126416}


wartosc wezla. 



Dokumentacja dla tej klasy została wygenerowana z pliku\-:\begin{DoxyCompactItemize}
\item 
C\-:/\-Users/\-Witek/\-Documents/\-Visual Studio 2012/\-Projects/\-Project47\-P\-A\-M\-S\-Ilab7/\-Project47\-P\-A\-M\-S\-Ilab7/\hyperlink{wezel_8h}{wezel.\-h}\end{DoxyCompactItemize}

\chapter{Dokumentacja plików}
\hypertarget{drzewo_8cpp}{\section{Dokumentacja pliku C\-:/\-Users/\-Witek/\-Documents/\-Visual Studio 2012/\-Projects/\-Project47\-P\-A\-M\-S\-Ilab7/\-Project47\-P\-A\-M\-S\-Ilab7/drzewo.cpp}
\label{drzewo_8cpp}\index{C\-:/\-Users/\-Witek/\-Documents/\-Visual Studio 2012/\-Projects/\-Project47\-P\-A\-M\-S\-Ilab7/\-Project47\-P\-A\-M\-S\-Ilab7/drzewo.\-cpp@{C\-:/\-Users/\-Witek/\-Documents/\-Visual Studio 2012/\-Projects/\-Project47\-P\-A\-M\-S\-Ilab7/\-Project47\-P\-A\-M\-S\-Ilab7/drzewo.\-cpp}}
}
{\ttfamily \#include $<$iostream$>$}\\*
{\ttfamily \#include \char`\"{}drzewo.\-h\char`\"{}}\\*
{\ttfamily \#include \char`\"{}wezel.\-h\char`\"{}}\\*

\hypertarget{drzewo_8h}{\section{Dokumentacja pliku C\-:/\-Users/\-Witek/\-Documents/\-Visual Studio 2012/\-Projects/\-Project47\-P\-A\-M\-S\-Ilab7/\-Project47\-P\-A\-M\-S\-Ilab7/drzewo.h}
\label{drzewo_8h}\index{C\-:/\-Users/\-Witek/\-Documents/\-Visual Studio 2012/\-Projects/\-Project47\-P\-A\-M\-S\-Ilab7/\-Project47\-P\-A\-M\-S\-Ilab7/drzewo.\-h@{C\-:/\-Users/\-Witek/\-Documents/\-Visual Studio 2012/\-Projects/\-Project47\-P\-A\-M\-S\-Ilab7/\-Project47\-P\-A\-M\-S\-Ilab7/drzewo.\-h}}
}
{\ttfamily \#include \char`\"{}wezel.\-h\char`\"{}}\\*
{\ttfamily \#include $<$iostream$>$}\\*
\subsection*{Komponenty}
\begin{DoxyCompactItemize}
\item 
class \hyperlink{class_drzewo}{Drzewo}
\begin{DoxyCompactList}\small\item\em Klasa \hyperlink{class_drzewo}{Drzewo} przedstawia drzewo binarne. Sklada sie z wielu wezlow. \end{DoxyCompactList}\end{DoxyCompactItemize}

\hypertarget{hashen_8cpp}{\section{Dokumentacja pliku C\-:/\-Users/\-Witek/\-Documents/\-Visual Studio 2012/\-Projects/\-Project47\-P\-A\-M\-S\-Ilab7/\-Project47\-P\-A\-M\-S\-Ilab7/hashen.cpp}
\label{hashen_8cpp}\index{C\-:/\-Users/\-Witek/\-Documents/\-Visual Studio 2012/\-Projects/\-Project47\-P\-A\-M\-S\-Ilab7/\-Project47\-P\-A\-M\-S\-Ilab7/hashen.\-cpp@{C\-:/\-Users/\-Witek/\-Documents/\-Visual Studio 2012/\-Projects/\-Project47\-P\-A\-M\-S\-Ilab7/\-Project47\-P\-A\-M\-S\-Ilab7/hashen.\-cpp}}
}
{\ttfamily \#include $<$iostream$>$}\\*
{\ttfamily \#include \char`\"{}hashen.\-h\char`\"{}}\\*

\hypertarget{hashen_8h}{\section{Dokumentacja pliku C\-:/\-Users/\-Witek/\-Documents/\-Visual Studio 2012/\-Projects/\-Project47\-P\-A\-M\-S\-Ilab7/\-Project47\-P\-A\-M\-S\-Ilab7/hashen.h}
\label{hashen_8h}\index{C\-:/\-Users/\-Witek/\-Documents/\-Visual Studio 2012/\-Projects/\-Project47\-P\-A\-M\-S\-Ilab7/\-Project47\-P\-A\-M\-S\-Ilab7/hashen.\-h@{C\-:/\-Users/\-Witek/\-Documents/\-Visual Studio 2012/\-Projects/\-Project47\-P\-A\-M\-S\-Ilab7/\-Project47\-P\-A\-M\-S\-Ilab7/hashen.\-h}}
}
{\ttfamily \#include $<$iostream$>$}\\*
{\ttfamily \#include $<$string$>$}\\*
\subsection*{Komponenty}
\begin{DoxyCompactItemize}
\item 
class \hyperlink{class_hash_en}{Hash\-En}
\begin{DoxyCompactList}\small\item\em Klasa \hyperlink{class_hash_en}{Hash\-En} przedstawia element mapy tablicy haszujacej. \end{DoxyCompactList}\end{DoxyCompactItemize}

\hypertarget{hashm_8cpp}{\section{Dokumentacja pliku C\-:/\-Users/\-Witek/\-Documents/\-Visual Studio 2012/\-Projects/\-Project47\-P\-A\-M\-S\-Ilab7/\-Project47\-P\-A\-M\-S\-Ilab7/hashm.cpp}
\label{hashm_8cpp}\index{C\-:/\-Users/\-Witek/\-Documents/\-Visual Studio 2012/\-Projects/\-Project47\-P\-A\-M\-S\-Ilab7/\-Project47\-P\-A\-M\-S\-Ilab7/hashm.\-cpp@{C\-:/\-Users/\-Witek/\-Documents/\-Visual Studio 2012/\-Projects/\-Project47\-P\-A\-M\-S\-Ilab7/\-Project47\-P\-A\-M\-S\-Ilab7/hashm.\-cpp}}
}
{\ttfamily \#include $<$iostream$>$}\\*
{\ttfamily \#include \char`\"{}hashm.\-h\char`\"{}}\\*

\hypertarget{hashm_8h}{\section{Dokumentacja pliku C\-:/\-Users/\-Witek/\-Documents/\-Visual Studio 2012/\-Projects/\-Project47\-P\-A\-M\-S\-Ilab7/\-Project47\-P\-A\-M\-S\-Ilab7/hashm.h}
\label{hashm_8h}\index{C\-:/\-Users/\-Witek/\-Documents/\-Visual Studio 2012/\-Projects/\-Project47\-P\-A\-M\-S\-Ilab7/\-Project47\-P\-A\-M\-S\-Ilab7/hashm.\-h@{C\-:/\-Users/\-Witek/\-Documents/\-Visual Studio 2012/\-Projects/\-Project47\-P\-A\-M\-S\-Ilab7/\-Project47\-P\-A\-M\-S\-Ilab7/hashm.\-h}}
}
{\ttfamily \#include $<$iostream$>$}\\*
{\ttfamily \#include \char`\"{}hashen.\-h\char`\"{}}\\*
\subsection*{Komponenty}
\begin{DoxyCompactItemize}
\item 
class \hyperlink{class_hash_m}{Hash\-M}
\begin{DoxyCompactList}\small\item\em Klasa \hyperlink{class_hash_m}{Hash\-M} przechowuje w sobie elementy (klucze i wartosci) w tablicy mieszajacej. \end{DoxyCompactList}\end{DoxyCompactItemize}

\hypertarget{main_8cpp}{\section{Dokumentacja pliku C\-:/\-Users/\-Witek/\-Documents/\-Visual Studio 2012/\-Projects/\-Project48\-P\-A\-M\-S\-Ilab8/\-Project48\-P\-A\-M\-S\-Ilab8/main.cpp}
\label{main_8cpp}\index{C\-:/\-Users/\-Witek/\-Documents/\-Visual Studio 2012/\-Projects/\-Project48\-P\-A\-M\-S\-Ilab8/\-Project48\-P\-A\-M\-S\-Ilab8/main.\-cpp@{C\-:/\-Users/\-Witek/\-Documents/\-Visual Studio 2012/\-Projects/\-Project48\-P\-A\-M\-S\-Ilab8/\-Project48\-P\-A\-M\-S\-Ilab8/main.\-cpp}}
}
{\ttfamily \#include \char`\"{}operacja.\-h\char`\"{}}\\*
\subsection*{Funkcje}
\begin{DoxyCompactItemize}
\item 
int \hyperlink{main_8cpp_ae66f6b31b5ad750f1fe042a706a4e3d4}{main} ()
\begin{DoxyCompactList}\small\item\em Funkcja main wykonujaca wyszukiwanie sciezek w grafie i liczaca czasy. \end{DoxyCompactList}\end{DoxyCompactItemize}


\subsection{Dokumentacja funkcji}
\hypertarget{main_8cpp_ae66f6b31b5ad750f1fe042a706a4e3d4}{\index{main.\-cpp@{main.\-cpp}!main@{main}}
\index{main@{main}!main.cpp@{main.\-cpp}}
\subsubsection[{main}]{\setlength{\rightskip}{0pt plus 5cm}int main (
\begin{DoxyParamCaption}
{}
\end{DoxyParamCaption}
)}}\label{main_8cpp_ae66f6b31b5ad750f1fe042a706a4e3d4}


Funkcja main wykonujaca wyszukiwanie sciezek w grafie i liczaca czasy. 

W funkcji main wykonywane sa nastepujace operacje\-:


\begin{DoxyItemize}
\item Tworzony jest obiekt klasy \hyperlink{class_operacja}{Operacja}
\item Uruchamiany jest interfejs i mozliwe jest wybranie algorytmu przeszukiwania grafu, ktory chcemy obserwowac. 
\end{DoxyItemize}
\hypertarget{operacja_8cpp}{\section{Dokumentacja pliku C\-:/\-Users/\-Witek/\-Documents/\-Visual Studio 2012/\-Projects/\-Project48\-P\-A\-M\-S\-Ilab8/\-Project48\-P\-A\-M\-S\-Ilab8/operacja.cpp}
\label{operacja_8cpp}\index{C\-:/\-Users/\-Witek/\-Documents/\-Visual Studio 2012/\-Projects/\-Project48\-P\-A\-M\-S\-Ilab8/\-Project48\-P\-A\-M\-S\-Ilab8/operacja.\-cpp@{C\-:/\-Users/\-Witek/\-Documents/\-Visual Studio 2012/\-Projects/\-Project48\-P\-A\-M\-S\-Ilab8/\-Project48\-P\-A\-M\-S\-Ilab8/operacja.\-cpp}}
}
{\ttfamily \#include \char`\"{}operacja.\-h\char`\"{}}\\*
{\ttfamily \#include $<$string$>$}\\*

\hypertarget{operacja_8h}{\section{Dokumentacja pliku C\-:/\-Users/\-Witek/\-Documents/\-Visual Studio 2012/\-Projects/\-Project43\-P\-A\-M\-S\-Ilab2/\-Project43\-P\-A\-M\-S\-Ilab2/operacja.h}
\label{operacja_8h}\index{C\-:/\-Users/\-Witek/\-Documents/\-Visual Studio 2012/\-Projects/\-Project43\-P\-A\-M\-S\-Ilab2/\-Project43\-P\-A\-M\-S\-Ilab2/operacja.\-h@{C\-:/\-Users/\-Witek/\-Documents/\-Visual Studio 2012/\-Projects/\-Project43\-P\-A\-M\-S\-Ilab2/\-Project43\-P\-A\-M\-S\-Ilab2/operacja.\-h}}
}
{\ttfamily \#include $<$iostream$>$}\\*
{\ttfamily \#include $<$fstream$>$}\\*
{\ttfamily \#include $<$ctime$>$}\\*
\subsection*{Komponenty}
\begin{DoxyCompactItemize}
\item 
class \hyperlink{class_operacja}{Operacja}
\begin{DoxyCompactList}\small\item\em Deklaracja klasy \hyperlink{class_operacja}{Operacja}. \end{DoxyCompactList}\end{DoxyCompactItemize}
\subsection*{Funkcje}
\begin{DoxyCompactItemize}
\item 
void \hyperlink{operacja_8h_a4f45ead62a93f12bf227fae7303fb5d4}{Wyswietl\-Tablice} (int tab\mbox{[}$\,$\mbox{]}, int Ilosc)
\end{DoxyCompactItemize}


\subsection{Dokumentacja funkcji}
\hypertarget{operacja_8h_a4f45ead62a93f12bf227fae7303fb5d4}{\index{operacja.\-h@{operacja.\-h}!Wyswietl\-Tablice@{Wyswietl\-Tablice}}
\index{Wyswietl\-Tablice@{Wyswietl\-Tablice}!operacja.h@{operacja.\-h}}
\subsubsection[{Wyswietl\-Tablice}]{\setlength{\rightskip}{0pt plus 5cm}void Wyswietl\-Tablice (
\begin{DoxyParamCaption}
\item[{int}]{tab\mbox{[}$\,$\mbox{]}, }
\item[{int}]{Ilosc}
\end{DoxyParamCaption}
)}}\label{operacja_8h_a4f45ead62a93f12bf227fae7303fb5d4}
Funkcja pomocnicza wyswietlajaca wyrazy dowolnej tablicy w zaleznosci od ilosci jej elementow.

Funkcja wyswietlajaca nie tylko tablice z klasy \hyperlink{class_operacja}{Operacja}, lecz kazda inna o zadanej ilosci elementow Argumenty i najwazniejsze pola funkcji


\begin{DoxyItemize}
\item tab\mbox{[}\mbox{]} -\/ tablica, ktora jest wyswietlana
\item Ilosc -\/ ilosc danych do wyswietlenia 
\end{DoxyItemize}
\hypertarget{wezel_8h}{\section{Dokumentacja pliku C\-:/\-Users/\-Witek/\-Documents/\-Visual Studio 2012/\-Projects/\-Project47\-P\-A\-M\-S\-Ilab7/\-Project47\-P\-A\-M\-S\-Ilab7/wezel.h}
\label{wezel_8h}\index{C\-:/\-Users/\-Witek/\-Documents/\-Visual Studio 2012/\-Projects/\-Project47\-P\-A\-M\-S\-Ilab7/\-Project47\-P\-A\-M\-S\-Ilab7/wezel.\-h@{C\-:/\-Users/\-Witek/\-Documents/\-Visual Studio 2012/\-Projects/\-Project47\-P\-A\-M\-S\-Ilab7/\-Project47\-P\-A\-M\-S\-Ilab7/wezel.\-h}}
}
{\ttfamily \#include $<$iostream$>$}\\*
{\ttfamily \#include $<$string$>$}\\*
\subsection*{Komponenty}
\begin{DoxyCompactItemize}
\item 
class \hyperlink{class_wezel}{Wezel}
\begin{DoxyCompactList}\small\item\em Klasa \hyperlink{class_wezel}{Wezel} opisuje obiekt bedacy elementem wiekszej klasy \hyperlink{class_drzewo}{Drzewo} -\/ drzewa binarnego. Posiada on swoj klucz -\/ string, oraz wartosc -\/ int. \end{DoxyCompactList}\end{DoxyCompactItemize}

\printindex
\end{document}
