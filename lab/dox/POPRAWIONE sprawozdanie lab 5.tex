%\documentclass{article}
\documentclass[12pt,a4paper,titlepage]{article}
\usepackage{graphicx}
\usepackage{graphics}
\usepackage{epsfig}
\usepackage{amsmath}
\usepackage{amssymb}
\usepackage{amsthm}
\usepackage{booktabs}
\usepackage{stmaryrd}
\usepackage{url}
\usepackage{longtable}
\usepackage[figuresright]{rotating}
\usepackage[utf8]{inputenc}
\usepackage[T1]{fontenc}
\usepackage[polish]{babel}
\usepackage{geometry}
\usepackage{pslatex}
\usepackage{ulem}
\usepackage{lipsum}
\usepackage{listings}
\usepackage{url}
\usepackage{Here}
\usepackage{color}
\usepackage[ruled,vlined,linesnumbered]{algorithm2e}
\selectlanguage{polish}
\definecolor{szary}{gray}{0.6}
\setlength{\textwidth}{400pt}
\lstset{numbers=left, numberstyle=\tiny, basicstyle=\scriptsize\ttfamily, breaklines=true, captionpos=b, tabsize=2}

\makeindex

\title{Laboratorium PAiMSI 5\newline
Poprawione sprawozdanie}
\date{\today}
\author{Witold Zimnicki - nr 200465}



\usepackage{pgfplots}
\usepackage{filecontents}
\begin{filecontents*}{data1.csv}
a,b,c,d
10,0,0,0
100,0,0,0
1000,1,0,0
5000,10,0,1
10000,27,1,2
50000,213,6,9
100000,483,14,17
\end{filecontents*}

\begin{filecontents*}{data2.csv}
a,b,c,d
10,0,0,0
100,0,0,0
1000,0,0,0
5000,0,0,1
10000,1,1,2
50000,6,6,9
100000,12,14,17
500000,76,75,93
1000000,187,4,4
2000000,523,3,4
5000000,2519,3,4
\end{filecontents*}


\begin{filecontents*}{data3.csv}
a,b,c,d
10,0,0,0
100,0,0,0
1000,0,0,0
5000,0,0,1
10000,1,1,2
50000,8,6,9
100000,18,14,17
500000,94,75,93
1000000,193,4,4
2000000,396,3,4
5000000,1008,3,4
\end{filecontents*}
\begin{document}

	\maketitle
	\pagestyle{empty}
	\pagestyle{headings}
	
	\textbf{W sprawozdaniu znajdują się 3 wykresy:}\newline
	
	1. Posortowanie tablicy algorytmem quicksort.(Wariant pesymistyczny) \newline
	 
	2. Posortowanie tablicy algorytmem quicksort.(Wariant przeciętny ) \newline
	
	3. Posortowanie tablicy algorytmem quicksort.(Wariant optymistyczny) \newline
	
	
	
	Każda wartość (czas) dla danego rozmiaru problemu jest średnią czasów zmierzonych w pętli wykonującej się 10-krotnie dla danego sortowania. \newline
	\newline
	\newline
	\newpage
	

	\begin{tikzpicture}
		\begin{axis}[
			title=\textbf{QUICKSORT-przypadek pesymistyczny},
			xlabel= N - rozmiar problemu,
			ylabel=czas w ms,
			xlabel style={sloped like x axis},
			ylabel style={sloped}
			]
			\addplot table [x=a, y=b, col sep=comma] {data1.csv};


		\end{axis}
	\end{tikzpicture}
		\newline \newline
		\begin{tikzpicture}
		\begin{axis}[
			title=\textbf{QUICKSORT-przypadek przeciętny},
			xlabel= N - rozmiar problemu,
			ylabel=czas w ms,
			xlabel style={sloped like x axis},
			ylabel style={sloped}
			]
			\addplot table [x=a, y=b, col sep=comma] {data2.csv};
			


		\end{axis}
	\end{tikzpicture}
	\newline \newline
		\begin{tikzpicture}
		\begin{axis}[
			title=\textbf{QUICKSORT-przypadek optymistyczny},
			xlabel= N - rozmiar problemu,
			ylabel=czas w ms,
			xlabel style={sloped like x axis},
			ylabel style={sloped}
			]
			\addplot table [x=a, y=b, col sep=comma] {data3.csv};
			


		\end{axis}
	\end{tikzpicture}
	\newline \newline \newline
\textbf{Wnioski: \newline	\newline}
\textbf{-} Przypadek pesymistyczny potwierdza swoją największą złożoność czasową i powoduje kwadratowo rosnące wartości czasu sortowania. \newline
\newline
 \textbf{-} Przypadek przeciętny algorytmu jest efektywniejszy od optymistycznego dla stosunkowo małych rozmiarów problemów. Za to dla większych problemów, czasy wykonywania się sortowania są dużo niższe w wariancie optymistycznym. \newline
 \newline
 - W przypadku pesymistycznym dla większych rozmiarów N problemu niż na wykresie, przy kompilacji pojawiały się błędy.
 


\end{document}