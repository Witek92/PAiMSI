%\documentclass{article}
\documentclass[12pt,a4paper,titlepage]{article}
\usepackage{graphicx}
\usepackage{graphics}
\usepackage{epsfig}
\usepackage{amsmath}
\usepackage{amssymb}
\usepackage{amsthm}
\usepackage{booktabs}
\usepackage{stmaryrd}
\usepackage{url}
\usepackage{longtable}
\usepackage[figuresright]{rotating}
\usepackage[utf8]{inputenc}
\usepackage[T1]{fontenc}
\usepackage[polish]{babel}
\usepackage{geometry}
\usepackage{pslatex}
\usepackage{ulem}
\usepackage{lipsum}
\usepackage{listings}
\usepackage{url}
\usepackage{Here}
\usepackage{color}
\usepackage[ruled,vlined,linesnumbered]{algorithm2e}
\selectlanguage{polish}
\definecolor{szary}{gray}{0.6}
\setlength{\textwidth}{400pt}
\setlength{\parindent}{1pt}
\lstset{numbers=left, numberstyle=\tiny, basicstyle=\scriptsize\ttfamily, breaklines=true, captionpos=b, tabsize=2}

\makeindex
\frenchspacing
\title{Laboratorium PAiMSI 10 \\
Najdłuższy Wspólny Podciąg}
\date{25 maja 2014}
\author{Witold Zimnicki - nr 200465}


\begin{document}

	\maketitle
	\pagestyle{empty}
	\pagestyle{headings}
	\begin{large}
	\textbf{1. Cel ćwiczenia:}
	\newline
	\end{large}

	Celem ćwiczenia było zapoznanie się z algorytmem \textbf{NWP} (Najdłuższy Wspólny Podciąg) oraz jego zaimplementowanie w taki sposób, aby możliwe było zaprezentowanie wyników udowadniających jego poprawne działanie.\newline
	\newline

	
	
	\begin{large}
	\textbf{2. Najdłuższy Wspólny Podciąg:}
	\newline
	\end{large}
	
	\begin{itemize}
\item[-] \hspace{8pt}  Zadaniem algorytmu \textbf{NWP} jest wyszukanie najdłuższego wspólnego łańcucha znaków w dwóch ciągach wyrazów. Unikatowość swoją zawdzięcza temu, że elementy łańcucha nie muszą leżeć obok siebie. Wykorzystywany może być do wykrywania plagiatów lub do różnego rodzaju systemów kontroli wersji. \\
\item[-] \hspace{8pt} Problem NWP dwóch ciągów A i B o długościach odpowiednio n i m może być rozwiązany za pomocą metody programowania dynamicznego.
Algorytm ten ma złożoność obliczeniową rzędu $\mathcal{O}(n*m)$, gdzie n oraz m to długości ciągów A i B. Algorytm ten tworzy tablicę dwuwymiarową C (n na m) taką, że wartość C[i][j] jest długością NWP podciągów A[1..i] i B[1..j]. A więc po zakończeniu wypełniania tablicy C komórka C[n][m] będzie zawierała wartość będąca długością NWP oryginalnych ciągów wejściowych A i B.\\\newline
\newline
	\end{itemize}
	
	\begin{large}
	\textbf{Odtworzenie najdłuższego podciągu: }
	\newline
	\end{large}
	
	Polega ono na utworzeniu macierzy o wymiarach n+1 x m+1 (n i m to długości ciągów A i B). Pierwsza kolumna oraz wiersz są zerami. Następnie wypełniamy macierz aż do końca jednym schematem: 
	\begin{enumerate}
	\item Jeśli znak kolejnego wiersza i kolumny są takie same - inkrementowana jest wartość na ukos z lewej strony.
	\item Jeśli znaki kolejnego wiersza i kolumny nie są takie same - w zależności od tego która jest większa, przypisywana jest wartość sąsiada na górze, lub po lewej.
	\end{enumerate}
	
	\newpage
	
	\begin{large}
	\textbf{Sprawdzenie zawartości najdłuższego wspólnego podciągu: }
	\newline
	\end{large}
	
	Aby dowiedzieć się jaki to podciąg należy 'przejść' z ostatniej komórki macierzy do komórki początkowej według następującego algorytmu:
\begin{enumerate}
\item jeśli któraś komórka sąsiednia do tej w której jesteśmy (z lewej strony lub u góry) ma wartość identyczną, to przechodzimy do niej.
\item jeśli jesteśmy w komórce zerowej to zakończ.
\item jeśli nie ma takiej, do ciągu odpowiedzi dopisujemy na początek literę odpowiadającą tej komórce a następnie idziemy do komórki o wartości mniejszej o 1 (po skosie do góry i w lewo).
\item jeśli nie jesteśmy w komórce [0][0] to przeskocz do punktu 1. w przeciwnym wypadku Zakończ.
\end{enumerate}

W związku z tym, że w każdym kolejnym kroku można iść zarówno w lewo jak i do góry, można uzyskać dwa, lub więcej podciągów.\newline

\newpage
	\begin{large}
	\textbf{3. Wyniki działania: }\\
	\newline
	\end{large}
	\begin{itemize}
	\item[-] \textbf{Wariant z jednym możliwym najdłuższym wspólnym podciągiem:}
	\\
	\newline
	\hspace{5pt} Działanie zostało sprawdzone na przykładowych ciągach znaków: "\textit{\textbf{komputer}} " oraz "\textit{\textbf{komputerami}}". Ich najdłuższy wspólny podciąg to "\textbf{\color{red} komputer}". Sprawdzimy zatem ich działanie zaimplementowanego w ćwiczeniu algorytmu:
	\begin{figure}[H]
\centering
\includegraphics[width=1\linewidth]{./wariant1komputer}
\caption{Wariant 1}
\label{Wariant 1}
\end{figure}
	\newpage
	
	\item[-] \textbf{Wariant z dwoma możliwymi najdłuższymi wspólnymi podciągami (gdy idziemy od ostatniego elementu w górę, lub w lewo:}\\
	\newline
\hspace{5pt} Działanie zostało sprawdzone na przykładowych ciągach znaków: \textit{\textbf{giggiiggg}} oraz \textit{\textbf{igigi}}. Ich najdłuższy wspólny podciąg to \textbf{\color{red} igig} oraz \textbf{\color{red} gigi}. Sprawdzimy zatem ich działanie zaimplementowanego w ćwiczeniu algorytmu: \\
\newline
\begin{figure}[H]
\centering
\includegraphics[width=1\linewidth]{./wariant2igig}
\caption{Wariant 2}
\label{Wariant 2}
\end{figure}
\newpage
\item[-] \textbf{Wariant z brakiem wspólnego podciągu:}\\
\newline
\hspace{5pt} Działanie zostało sprawdzone na przykładowych ciągach znaków: \textit{\textbf{abcdef}} oraz \textit{\textbf{ghijkl}}. Sprawdzimy działanie zaimplementowanego w ćwiczeniu algorytmu:
\begin{figure}[H]
\centering
\includegraphics[width=1\linewidth]{./wariant3brak}
\caption{Wariant 3}
\label{Wariant 3}
\end{figure}
\end{itemize}
	
	
	
\begin{large}
	\textbf{4. Wnioski: \\}
	\newline
	\end{large}
\textbf{-} Wyniki testu potwierdzają poprawność działania zaimplementowanych funkcji \textbf{NWP}.
\begin{flushleft}
\textbf{-} Ćwiczenie pokazuje, że algorytm NWP może być przydatny w kontrolowaniu wersji dokumentów, książek czy też wykrywaniu plagiatów. \newline
 
\textbf{-} Przy dużych rozmiarach ciągów uwidacznia się stosunkowo duża złożoność czasowa algorytmu $\mathcal{O}(n*m)$, gdzie n i m to długości zadanych ciągów znaków .\newline
 
\textbf{-} Algorytm wyszukuje wspólny najdłuższy podciąg, nawet jeśli elementy ciągów zadanych nie leżą obok siebie. Nie wolno go zatem mylić z algorytmem \textit{ najdłuższego wspólnego podłańcucha}. \newline
 \end{flushleft} 

\end{document}