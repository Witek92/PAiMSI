%\documentclass{article}
\documentclass[12pt,a4paper,titlepage]{article}
\usepackage{graphicx}
\usepackage{graphics}
\usepackage{epsfig}
\usepackage{amsmath}
\usepackage{amssymb}
\usepackage{amsthm}
\usepackage{booktabs}
\usepackage{stmaryrd}
\usepackage{url}
\usepackage{longtable}
\usepackage[figuresright]{rotating}
\usepackage[utf8]{inputenc}
\usepackage[T1]{fontenc}
\usepackage[polish]{babel}
\usepackage{geometry}
\usepackage{pslatex}
\usepackage{ulem}
\usepackage{lipsum}
\usepackage{listings}
\usepackage{url}
\usepackage{Here}
\usepackage{color}
\usepackage[ruled,vlined,linesnumbered]{algorithm2e}
\selectlanguage{polish}
\definecolor{szary}{gray}{0.6}
\setlength{\textwidth}{400pt}
\lstset{numbers=left, numberstyle=\tiny, basicstyle=\scriptsize\ttfamily, breaklines=true, captionpos=b, tabsize=2}

\makeindex

\title{Laboratorium PAiMSI 9}
\date{18 maja 2014}
\author{Witold Zimnicki - nr 200465}



\usepackage{pgfplots}
\usepackage{filecontents}


\begin{filecontents*}{data2.csv}
a,b,c,d
5,2
10,4,0,0
20,10
40,38
60,99
80,219
100,369,0,0
120, 611
140, 925
160, 1378
180, 1854
200, 2698
220, 3246
240, 4093
250, 4871
\end{filecontents*}




\begin{document}

	\maketitle
	\pagestyle{empty}
	\pagestyle{headings}
	
	Sprawozdanie przedstawia wykres z zależnościami czasu od wielkości problemu dla szukania ścieżek \textbf{grafu nieskierowanego} metodą \textbf{Branch and Bound} (metoda podziału i ograniczeń). Graf zaimplementowany został jako tablica dwuwymiarowa (macierz sąsiedztwa) z wymiarami w x w, gdzie w to ilość wierzchołków grafu. Przed wykonaniami algorytmów przeszukiwania ścieżek, do grafu zostają dodane losowe wierzchołki(typu string) w ilości, która podawana jest na wejście standardowe. Szukana jest zawsze droga \textbf{od pierwszego wierzchołka, do ostatniego dodanego wierzchołka} (teoretycznie zawsze najbardziej złożona; dla 'bliższych' wierzchołków czasy były dużo mniejsze). Wybierana jest również ilość wykonań algorytmu tak, aby uzyskać dokładniejszą wartość czasową.(W przypadku tego ćwiczenia było to 10.)\newline
	\newline
	
	
	
		\begin{tikzpicture}
		\begin{axis}[
			title=\textbf{B and B},
			xlabel= N - rozmiar problemu,
			ylabel=czas w ms,
			xlabel style={sloped like x axis},
			ylabel style={sloped}
			]
			\addplot table [x=a, y=b, col sep=comma] {data2.csv};


		\end{axis}
	\end{tikzpicture}
	
	
\textbf{Wnioski: \newline	\newline}
\textbf{-} Porównanie algorytmu Branch And Bound z Breadth First Search i Depth First Search w wykonanych przeze mnie ćwiczeniach wykazało dużo mniejszą złożoność czasową tych algorytmów, co jest \textbf{mocno sprzeczne} z teorią. (dlatego nie widziano sensu zestawiania bezpośrednio ich wykresów w sprawozdaniu) \newline
\newline
 \textbf{-} Przyczyną może być inna implementacja grafu w poprzednim ćwiczeniu i inne wierzchołki, między którymi szukane były ścieżki.  \newline
 \newline
 \textbf{-} Wykres szukania ścieżek w grafie metodą Branch and Bound wykazał w wykonanym ćwiczeniu kwadratową złożoność czasową. 

\end{document}