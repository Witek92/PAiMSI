%\documentclass{article}
\documentclass[12pt,a4paper,titlepage]{article}
\usepackage{graphicx}
\usepackage{graphics}
\usepackage{epsfig}
\usepackage{amsmath}
\usepackage{amssymb}
\usepackage{amsthm}
\usepackage{booktabs}
\usepackage{stmaryrd}
\usepackage{url}
\usepackage{longtable}
\usepackage[figuresright]{rotating}
\usepackage[utf8]{inputenc}
\usepackage[T1]{fontenc}
\usepackage[polish]{babel}
\usepackage{geometry}
\usepackage{pslatex}
\usepackage{ulem}
\usepackage{lipsum}
\usepackage{listings}
\usepackage{url}
\usepackage{Here}
\usepackage{color}
\usepackage[ruled,vlined,linesnumbered]{algorithm2e}
\selectlanguage{polish}
\definecolor{szary}{gray}{0.6}
\setlength{\textwidth}{400pt}
\lstset{numbers=left, numberstyle=\tiny, basicstyle=\scriptsize\ttfamily, breaklines=true, captionpos=b, tabsize=2}

\makeindex

\title{Laboratorium PAiMSI 7}
\date{\today}
\author{Witold Zimnicki - nr 200465}



\usepackage{pgfplots}
\usepackage{filecontents}
\begin{filecontents*}{data1.csv}
a,b,c,d
10,0,0,0
100,1,2,0
1000,10,24,0
3000,60,60,3
5000,153,101,1
6000,217,128,4
8000,368,173,45
10000,563,197,2
20000,3385,427,4
30000,10907,617,4
40000,22097,922,4
50000,38556,1053,9

\end{filecontents*}




\begin{document}

	\maketitle
	\pagestyle{empty}
	\pagestyle{headings}
	
	Sprawozdanie przedstawia wykresy z zależnościami czasu od ilości danych dla wypełniania drzewa binarnego oraz tablicy mieszającej. Do tych struktur danych dodawane są losowe pary: kluczy oraz wartości.
	\textbf{W sprawozdaniu znajdują się 2 wykresy:}\newline
	
	1. Czasy wypełniania drzewa binarnego dla różnych ilości danych. \newline
	 
	2. Czasy wypełniania tablicy mieszającej dla różnych ilości danych. \newline
	
	
	

	\begin{tikzpicture}
		\begin{axis}[
			title=\textbf{Drzewo binarne},
			xlabel= N - rozmiar problemu,
			ylabel=czas w ms,
			xlabel style={sloped like x axis},
			ylabel style={sloped}
			]
			\addplot table [x=a, y=b, col sep=comma] {data1.csv};


		\end{axis}
	\end{tikzpicture}
		\newline
		\newline
		\begin{tikzpicture}
		\begin{axis}[
			title=\textbf{Tablica Mieszająca},
			xlabel= N - rozmiar problemu,
			ylabel=czas w ms,
			xlabel style={sloped like x axis},
			ylabel style={sloped}
			]
			\addplot table [x=a, y=c, col sep=comma] {data1.csv};
			


		\end{axis}
	\end{tikzpicture}
	
	
	
\textbf{Wnioski: \newline	\newline}
\textbf{-} Podsumowując, można stwierdzić, że zapełnianie drzewa binarnego przyjmuje postać zależności kwadratowej, natomiast tablicy mieszającej liniowej. \newline
\newline
 \textbf{-} Logiczną konsekwencją powyższego wniosku jest mniejsza złożoność czasowa struktury danych, jaką jest tablica mieszająca. \newline
 


\end{document}