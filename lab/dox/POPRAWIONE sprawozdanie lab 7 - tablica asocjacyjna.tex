%\documentclass{article}
\documentclass[12pt,a4paper,titlepage]{article}
\usepackage{graphicx}
\usepackage{graphics}
\usepackage{epsfig}
\usepackage{amsmath}
\usepackage{amssymb}
\usepackage{amsthm}
\usepackage{booktabs}
\usepackage{stmaryrd}
\usepackage{url}
\usepackage{longtable}
\usepackage[figuresright]{rotating}
\usepackage[utf8]{inputenc}
\usepackage[T1]{fontenc}
\usepackage[polish]{babel}
\usepackage{geometry}
\usepackage{pslatex}
\usepackage{ulem}
\usepackage{lipsum}
\usepackage{listings}
\usepackage{url}
\usepackage{Here}
\usepackage{color}
\usepackage[ruled,vlined,linesnumbered]{algorithm2e}
\selectlanguage{polish}
\definecolor{szary}{gray}{0.6}
\setlength{\textwidth}{400pt}
\lstset{numbers=left, numberstyle=\tiny, basicstyle=\scriptsize\ttfamily, breaklines=true, captionpos=b, tabsize=2}

\makeindex

\title{Poprawione sprawozdanie z laboratorium PAiMSI 7 -
Tablica Asocjacyjna}
\date{25 maja 2014}
\author{Witold Zimnicki - nr 200465}



\usepackage{pgfplots}
\usepackage{filecontents}
\begin{filecontents*}{data1.csv}
a,b,c,d
10,0,0,0
100,0,0,0
1000,1,1,1
5000,2,1,1
8000,3,2,1
10000,3,2,1
30000,8,8,1
50000,12,9,1
80000,12,10,1
100000,13,12,1

\end{filecontents*}




\begin{document}

	\maketitle
	\pagestyle{empty}
	\pagestyle{headings}
	\begin{large}
	\textbf{1. Cel ćwiczenia:}
	\newline
	\end{large}

	Celem ćwiczenia było przetestowanie czasów wyszukania wartości elementu według jego klucza w strukturach danych opartych na tablicy asocjacyjnej, takich jak:
	\begin{itemize}
\item[-] tablica asocjacyjna oparta na klasie \textit{vector}
\item[-] binarne drzewo przeszukiwań
\item[-] tablica mieszająca
\newline
\newline


	\end{itemize}
	
	
	\begin{large}
	\textbf{2. Wyniki testu:}
	\newline
	\end{large}
	
	Algorytmy wykonywane są następująco: \\
	Do podanych wyżej struktur danych zostaje dodana zadana ilość par wartości i kluczy (bez kolizji). Następnie pobierany jest uśredniony czas wyszukiwania wartości klucza dla losowo wybranych kluczy przy 20 powtórzeniach.\\
	\\
	
	
	\textbf{W sprawozdaniu znajdują się 3 wykresy:}\newline
	
1. Czasy szukania wartości klucza w tablicy asocjacyjnej dla różnych ilości danych. \newline
	 
2. Czasy szukania wartości klucza w tablicy mieszającej dla różnych ilości danych. \newline
	
3. Czasy szukania wartości klucza w drzewie BST dla różnych ilości danych. \newline
	
	\newpage
	
	\begin{large}
	\textbf{Wykresy:}
	\newline
	\end{large}
	
\begin{center}
\begin{tikzpicture}
		\begin{axis}[
			title=\textbf{TABLICA ASOCJACYJNA - \textit{vector}},
			xlabel= N - rozmiar problemu,
			ylabel=czas [ms],
			xlabel style={sloped like x axis},
			ylabel style={sloped}
			]
			\addplot table [x=a, y=b, col sep=comma] {data1.csv};

		\end{axis}
	\end{tikzpicture}
	
	
	
		\begin{tikzpicture}
		\begin{axis}[
			title=\textbf{DRZEWO BST},
			xlabel= N - rozmiar problemu,
			ylabel=czas [ms],
			xlabel style={sloped like x axis},
			ylabel style={sloped}
			]
			\addplot table [x=a, y=c, col sep=comma] {data1.csv};
			


		\end{axis}
	\end{tikzpicture}
	
	
		
		\begin{tikzpicture}
		\begin{axis}[
			title=\textbf{TABLICA MIESZAJĄCA},
			xlabel= N - rozmiar problemu,
			ylabel=czas  [ms],
			xlabel style={sloped like x axis},
			ylabel style={sloped}
			]
			\addplot table [x=a, y=d, col sep=comma] {data1.csv};
			


		\end{axis}
	\end{tikzpicture}
\end{center}
	
	
	
\begin{large}
	\textbf{3. Wnioski: \\}
	\newline
	\end{large}
\textbf{-} Wyniki testu potwierdzają w przybliżeniu następujące złożoności czasowe poszczególnych algorytmów: \newline
\begin{itemize}
\item Tablica asocjacyjna zaimplementowana na klasie \textit{vector}: $\mathcal{O}(log(n))$ 
\item Drzewo BST (drzewo bliskie zrównoważenia) ze złożonością $\mathcal{O}(log(n))$ 
\item Tablica mieszająca przy nie za dużych zbiorach: $\mathcal{O}(1)$
\end{itemize}

\begin{flushleft}
\textbf{-}  Powyższe ćwiczenie pokazuje, że na czas wykonywania się algorytmu składa się również miejsce w strukturze danych, z którego wyszukiwana będzie wartość, a nie tylko rozmiar problemu.\newline

 
 
\textbf{-}  Przy dużych rozmiarach problemu, dodawanie dużej ilości par kluczy i wartości do tablicy asocjacyjnej opartej na klasie \textit{vector} jest czasochłonne, gdyż wykazuje zależność liniową.\newline
 
\textbf{-} Generalny czas dostępu do wartości w drzewie BST był nieco mniejszy niż dla tablicy asocjacyjnej.\newline
 \end{flushleft} 

\end{document}