%\documentclass{article}
\documentclass[12pt,a4paper,titlepage]{article}
\usepackage{graphicx}
\usepackage{graphics}
\usepackage{epsfig}
\usepackage{amsmath}
\usepackage{amssymb}
\usepackage{amsthm}
\usepackage{booktabs}
\usepackage{stmaryrd}
\usepackage{url}
\usepackage{longtable}
\usepackage[figuresright]{rotating}
\usepackage[utf8]{inputenc}
\usepackage[T1]{fontenc}
\usepackage[polish]{babel}
\usepackage{geometry}
\usepackage{pslatex}
\usepackage{ulem}
\usepackage{lipsum}
\usepackage{listings}
\usepackage{url}
\usepackage{Here}
\usepackage{color}
\usepackage[ruled,vlined,linesnumbered]{algorithm2e}
\selectlanguage{polish}
\definecolor{szary}{gray}{0.6}
\setlength{\textwidth}{400pt}
\lstset{numbers=left, numberstyle=\tiny, basicstyle=\scriptsize\ttfamily, breaklines=true, captionpos=b, tabsize=2}

\makeindex

\title{Laboratorium PAiMSI 8}
\date{\today}
\author{Witold Zimnicki - nr 200465}



\usepackage{pgfplots}
\usepackage{filecontents}
\begin{filecontents*}{data1.csv}
a,b,c,d
10,0,0,0
100,1,0,0
1000,10,7,8
3000,33,22,22
5000,55,37,37
6000,66,43,44
8000,116,78,72
10000,137,92,105
20000,297,195,206
30000,480,432,432
40000,812,589,657
50000,,,688

\end{filecontents*}




\begin{document}

	\maketitle
	\pagestyle{empty}
	\pagestyle{headings}
	
	Sprawozdanie przedstawia wykresy z zależnościami czasu od wielkości problemu dla szukania ścieżek \textbf{grafu nieskierowanego} metodami \textbf{Breadth First Search} (przeszukiwanie wszerz) oraz \textbf{Depth First Search} (przeszukiwanie w głąb). Graf zaimplementowany został jako wskaźnik (tablica) z listami typu int, które opisują otoczenie wierzchołków. Przed wykonaniami algorytmów przeszukiwania ścieżek, do grafu zostają dodane losowe liczby w ilości, która podawana jest na wejście standardowe. Wybierana jest również ilość wykonań algorytmu tak, aby uzyskać dokładniejszą wartość czasową.(W przypadku tego ćwiczenia było to 20.)\newline
	\newline
	\textbf{W sprawozdaniu znajdują się 3 wykresy:}\newline
	
	1. Czasy wyszukania ścieżek grafu metodą BFS. \newline
	 
	2. Czasy wyszukania ścieżek grafu metodą DFS (rozpatrując wszystkie wierzchołki). \newline
	
	3. Czasy wyszukania ścieżek grafu metodą DFS (z miejsca od losowo wybranego wierzchołka).
	
	
	

	\begin{tikzpicture}
		\begin{axis}[
			title=\textbf{BFS},
			xlabel= N - rozmiar problemu,
			ylabel=czas w ms,
			xlabel style={sloped like x axis},
			ylabel style={sloped}
			]
			\addplot table [x=a, y=d, col sep=comma] {data1.csv};


		\end{axis}
	\end{tikzpicture}
		\newline
		\newline
		\begin{tikzpicture}
		\begin{axis}[
			title=\textbf{DFS wszystkie wierzchołki},
			xlabel= N - rozmiar problemu,
			ylabel=czas w ms,
			xlabel style={sloped like x axis},
			ylabel style={sloped}
			]
			\addplot table [x=a, y=c, col sep=comma] {data1.csv};
			


		\end{axis}
	\end{tikzpicture}
	
	
		\begin{tikzpicture}
		\begin{axis}[
			title=\textbf{DFS od wylosowanego wierzchołka},
			xlabel= N - rozmiar problemu,
			ylabel=czas w ms,
			xlabel style={sloped like x axis},
			ylabel style={sloped}
			]
			\addplot table [x=a, y=d, col sep=comma] {data1.csv};
			


		\end{axis}
	\end{tikzpicture}
	
	
	
\textbf{Wnioski: \newline	\newline}
\textbf{-} Podsumowując, można stwierdzić, że w przypadku powyżej wykonywanych pomiarów, złożoność obliczeniowa wszystkich metod była zbliżona. \newline
\newline
 \textbf{-} Obserwując dokładnie, szukanie algorytmem Depth First Search po wszystkich wierzchołkach dawało stosunkowo nieco mniejsze czasy niż przy innych omawianych metodach. \newline
 


\end{document}