%\documentclass{article}
\documentclass[12pt,a4paper,titlepage]{article}
\usepackage{graphicx}
\usepackage{graphics}
\usepackage{epsfig}
\usepackage{amsmath}
\usepackage{amssymb}
\usepackage{amsthm}
\usepackage{booktabs}
\usepackage{stmaryrd}
\usepackage{url}
\usepackage{longtable}
\usepackage[figuresright]{rotating}
\usepackage[utf8]{inputenc}
\usepackage[T1]{fontenc}
\usepackage[polish]{babel}
\usepackage{geometry}
\usepackage{pslatex}
\usepackage{ulem}
\usepackage{lipsum}
\usepackage{listings}
\usepackage{url}
\usepackage{Here}
\usepackage{color}
\usepackage[ruled,vlined,linesnumbered]{algorithm2e}
\selectlanguage{polish}
\definecolor{szary}{gray}{0.6}
\setlength{\textwidth}{400pt}
\lstset{numbers=left, numberstyle=\tiny, basicstyle=\scriptsize\ttfamily, breaklines=true, captionpos=b, tabsize=2}

\makeindex

\title{Laboratorium PAiMSI 3
\newline
\newline
Poprawione sprawozdanie}
\date{\today}
\author{Witold Zimnicki - nr 200465}



\usepackage{pgfplots}
\usepackage{filecontents}
\begin{filecontents*}{data1.csv}
a,b,c,d
10,0,0,1
100,0,0,5
1000,1,33,1
5000,19,822,19
10000,48,2904,48
50000,48,91433,48
\end{filecontents*}

\begin{filecontents*}{data2.csv}
a,b,c,d
10,0,0,1
100,0,0,5
1000,6,0,1
5000,91,0,91
10000,210,1,210
50000,48,2,48
100000,48,4,48
500000,48,26,48
\end{filecontents*}

\begin{filecontents*}{data3.csv}
a,b,c,d
10,0,0,1
100,0,0,5
1000,0,34,0
5000,1,1011,1
10000,1,4868,1
50000,1,124101,1
\end{filecontents*}

\begin{filecontents*}{data4.csv}
a,b,c,d
10,0,0,1
100,0,0,5
1000,6,0,1
5000,91,0,91
10000,210,0,210
50000,48,2,48
100000,48,5,48
500000,48,18,48
\end{filecontents*}


\begin{document}

	\maketitle
	\pagestyle{empty}
	\pagestyle{headings}
	
	\textbf{W sprawozdaniu znajdują się 4 wykresy:}\newline
	
	1. Raport z zapełniania stosu zaimplementowanego jako tablica, przy czym przy
braku miejsca na stosie, jego rozmiar zwiększany jest o jeden. \newline
	 
	2. Raport z zapełniania stosu zaimplementowanego jako tablica, przy czym przy
braku miejsca na stosie, jego rozmiar zwiększany jest dwukrotnie. \newline
	
	3. Raport z zapełniania stosu zaimplementowanego jako lista. \newline
	
	4. Raport z zapełnienia kolejki zaimplementowanej tablicowo. \newline
	
	
	Każda wartość (czas) dla danego rozmiaru problemu jest średnią czasów zmierzonych w pętli wykonującej się 10-krotnie dla danego sortowania. \newline
	

	\begin{tikzpicture}
		\begin{axis}[
			title=\textbf{1. Stos Tablicowy (powiekszanie o jeden) \newline
			\newline},
			xlabel= N - rozmiar problemu,
			ylabel=czas w ms,
			xlabel style={sloped like x axis},
			ylabel style={sloped}
			]
			\addplot table [x=a, y=c, col sep=comma] {data1.csv};
			


		\end{axis}
	\end{tikzpicture}
	
	\begin{tikzpicture}
		\begin{axis}[
			title= \textbf{2. Stos Tablicowy (powiekszanie razy dwa)},
			xlabel= N - rozmiar problemu,
			ylabel=czas w ms,
			xlabel style={sloped like x axis},
			ylabel style={sloped}
			]
			\addplot table [x=a, y=c, col sep=comma] {data2.csv};
			


		\end{axis}
	\end{tikzpicture}
	
	\begin{tikzpicture}

		\begin{axis}[
			title= \textbf{3. Stos Listowy},
			xlabel= N - rozmiar problemu,
			ylabel=czas w ms,
			xlabel style={sloped like x axis},
			ylabel style={sloped}
			]
			\addplot table [x=a, y=c, col sep=comma] {data3.csv};
			


		\end{axis}


	\end{tikzpicture}
	
	\begin{tikzpicture}

		\begin{axis}[
			title= \textbf{4. Kolejka tablicowa},
			xlabel= N - rozmiar problemu,
			ylabel=czas w ms,
			xlabel style={sloped like x axis},
			ylabel style={sloped}
			]
			\addplot table [x=a, y=c, col sep=comma] {data4.csv};
			


		\end{axis}


	\end{tikzpicture}
\newline
\newline
\newline
\textbf{Wnioski: }\newline
\newline
- Rozpatrując dwa omówione sposoby powiększania stosu ('o jeden' oraz 'razy dwa') dużo mniejszą złożonością czasową wykazał się sposób 'razy dwa'.\newline
\newline
- Najefektywniejszym algorytmem było zapełnianie kolejki zaimplementowanej jako tablica (przy problemie N=500 000 średnia wykonywania się jego wynosiła 18 ms.\newline
\newline
- Stos zaimplementowany jako lista oraz stos zapełniający się z każdorazowym jego powiększeniem o jeden był efektywny tylko dla małych rozmiarów problemu. Dla większych - czas rósł w tempie kwadratowym.



\end{document}