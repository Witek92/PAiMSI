%\documentclass{article}
\documentclass[12pt,a4paper,titlepage]{article}
\usepackage{graphicx}
\usepackage{graphics}
\usepackage{epsfig}
\usepackage{amsmath}
\usepackage{amssymb}
\usepackage{amsthm}
\usepackage{booktabs}
\usepackage{stmaryrd}
\usepackage{url}
\usepackage{longtable}
\usepackage[figuresright]{rotating}
\usepackage[utf8]{inputenc}
\usepackage[T1]{fontenc}
\usepackage[polish]{babel}
\usepackage{geometry}
\usepackage{pslatex}
\usepackage{ulem}
\usepackage{lipsum}
\usepackage{listings}
\usepackage{url}
\usepackage{Here}
\usepackage{color}
\usepackage[ruled,vlined,linesnumbered]{algorithm2e}
\selectlanguage{polish}
\definecolor{szary}{gray}{0.6}
\setlength{\textwidth}{400pt}
\lstset{numbers=left, numberstyle=\tiny, basicstyle=\scriptsize\ttfamily, breaklines=true, captionpos=b, tabsize=2}

\makeindex

\title{Laboratorium PAiMSI 4
\newline Poprawione sprawozdanie}
\date{\today}
\author{Witold Zimnicki - nr 200465}



\usepackage{pgfplots}
\usepackage{filecontents}
\begin{filecontents*}{data1.csv}
a,b,c,d
10,0,0,1
100,0,0,5
1000,1,1,1
5000,19,10,19
10000,48,27,48
50000,48,213,48
100000,48,483,48
\end{filecontents*}

\begin{filecontents*}{data2.csv}
a,b,c,d
10,0,0,1
100,0,0,5
1000,6,4,1
5000,91,92,91
10000,210,471,210
\end{filecontents*}

\begin{filecontents*}{data3.csv}
a,b,c,d
10,0,0,1
100,0,0,5
1000,1,0,1
5000,19,1,19
10000,48,1,48
50000,48,8,48
100000,48,18,48
500000,3,102,49
1000000,1,213,43
\end{filecontents*}

\begin{filecontents*}{data4.csv}
a,b,c,d
10,0,0,1
100,0,0,5
1000,5,5,5
5000,125,121,125
10000,513,513,513
50000,3,14576,43
100000,3,58831,3

\end{filecontents*}


\begin{document}

	\maketitle
	\pagestyle{empty}
	\pagestyle{headings}
	
	\textbf{W sprawozdaniu znajdują się 4 wykresy:}\newline
	
	1. Posortowanie tablicy algorytmem quicksort.(złożoność czasowa $O(n^2))$ \newline
	 
	2. Posortowanie tablicy algorytmem mergesort.(złożoność czasowa $O(n*log (n)))$ \newline
	
	3. Posortowanie tablicy algorytmem heapsort.(złożoność czasowa $O(n*log (n)))$ \newline
	
	4. Posortowanie tablicy algorytmem sortowania bąbelkowego. \newline
					(złożoność czasowa $O(n^2))$ \newline
	
	
	Każda wartość (czas) dla danego rozmiaru problemu jest średnią czasów zmierzonych w pętli wykonującej się 10-krotnie dla danego sortowania. \newline
	

	\begin{tikzpicture}
		\begin{axis}[
			title=\textbf{1. Sortowanie quicksort},
			xlabel= N - rozmiar problemu,
			ylabel=czas w ms,
			xlabel style={sloped like x axis},
			ylabel style={sloped}
			]
			\addplot table [x=a, y=c, col sep=comma] {data1.csv};
			


		\end{axis}
	\end{tikzpicture}
	
	\begin{tikzpicture}
		\begin{axis}[
			title= \textbf{2. Sortowanie mergesort},
			xlabel= N - rozmiar problemu,
			ylabel=czas w ms,
			xlabel style={sloped like x axis},
			ylabel style={sloped}
			]
			\addplot table [x=a, y=c, col sep=comma] {data2.csv};
			


		\end{axis}
	\end{tikzpicture}
	
	\begin{tikzpicture}

		\begin{axis}[
			title= \textbf{3. Sortowanie heapsort},
			xlabel= N - rozmiar problemu,
			ylabel=czas w ms,
			xlabel style={sloped like x axis},
			ylabel style={sloped}
			]
			\addplot table [x=a, y=c, col sep=comma] {data3.csv};
			


		\end{axis}


	\end{tikzpicture}
	
	\begin{tikzpicture}

		\begin{axis}[
			title= \textbf{4. Sortowanie bąbelkowe},
			xlabel= N - rozmiar problemu,
			ylabel=czas w ms,
			xlabel style={sloped like x axis},
			ylabel style={sloped}
			]
			\addplot table [x=a, y=c, col sep=comma] {data4.csv};
			


		\end{axis}


	\end{tikzpicture}
\newline
\newline
\textbf{Wnioski: \newline}
\newline
- Wśród zaimplementowanych przeze mnie algorytmów sortowania, najefektywniejszym był 'heapsort'. Dla rozmiaru problemu N=1 000 000 średni czas wykonywania się algorytmu wynosił stosunkowo mało - 213 ms.\newline
\newline
- Wykresy efektywności algorytmów w dużym stopniu odzwierciedlają ich teoretyczne złożoności czasowe.\newline
\newline
- W przypadku quicksort oraz mergesort, przy próbie zwiększenia problemu występowały błędy.



\end{document}